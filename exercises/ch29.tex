\title{Some Proofs}

\documentclass[12pt]{article}
\usepackage{amsmath,amsthm,amssymb}
\usepackage[utf8]{inputenc}
    \newcommand\xqed[1]{%
        \leavevmode\unskip\penalty9999 \hbox{}\nobreak\hfill
        \quad\hbox{#1}}
    \newcommand\qedtri{\xqed{$\triangle$}}
\setlength\parindent{0pt}

\begin{document}
\subsection{B: Further Examples of Finite Extensions}
Let $F$ be a field of characteristic $\neq 2$. Let $a, b \in F$ and $a \neq b $.

\subsubsection{}
Prove that any field $F$ containing $\sqrt{a} + \sqrt{b}$ also contains $\sqrt{a}$ and $\sqrt{b}$.
Conclude that $F(\sqrt{a} + \sqrt{b}) = F(\sqrt{a}, \sqrt{b})$.

\begin{proof}
By definition, $\tau = \sqrt{a}+\sqrt{b} \in F(\sqrt{a} + \sqrt{b})$.
Since $F(\sqrt{a} + \sqrt{b})$ is a field, $\tau^2 = a+2\sqrt{ab}+b \in F(\sqrt{a} + \sqrt{b})$.
Then, it must be that $\sqrt{ab} \in F(\sqrt{a} + \sqrt{b})$.
Well, since the product of any two elements in a field is an element of the field,
it must be that $\tau\sqrt{ab} = a\sqrt{b}+b\sqrt{a} \in F(\sqrt{a} + \sqrt{b})$.
Hence, we have shown that any field $F$ containing $\sqrt{a} + \sqrt{b}$ also contains $\sqrt{a}$ and $\sqrt{b}$.

Since by definition, $F(\sqrt{a}, \sqrt{b})$ is the minimum field that contains
$\sqrt{a}$ and $\sqrt{b}$, it follows that $F(\sqrt{a}, \sqrt{b}) \subseteq F(\sqrt{a} + \sqrt{b})$.
The reverse inclusion is also true because $\sqrt{a}+\sqrt{b} \in F(\sqrt{a}, \sqrt{b})$
and $F(\sqrt{a}+\sqrt{b})$ is the minimum field containing $\sqrt{a} + \sqrt{b}$.
\end{proof}
\end{document}
