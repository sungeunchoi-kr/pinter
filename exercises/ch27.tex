\title{Some Proofs}

\documentclass[12pt]{article}
\usepackage{amsmath,amsthm,amssymb}
\usepackage[utf8]{inputenc}
    \newcommand\xqed[1]{%
        \leavevmode\unskip\penalty9999 \hbox{}\nobreak\hfill
        \quad\hbox{#1}}
    \newcommand\qedtri{\xqed{$\triangle$}}
\setlength\parindent{0pt}

\begin{document}
\subsection{F: Quadratic Extensions}
If $F$ is a field whose characteristic is $\neq 2$, any quadratic extension of $F$ is of the form
$F(\sqrt{a})$ for some $a \in F$.

\begin{proof}
Suppose $F(\alpha)$ is a quadratic extension of $F$. Then, by definition, there is a minimum
polynomial of $\alpha$ over $F$ of degree 2, $g(x)$, of the form
$$g(x) = x^2 + \beta x + \gamma \text{ where } g(\alpha)=0 \text{ and } \beta \text{, } \gamma \in F \text{.}$$
By completing the square,
$$g(x) = \left(x + \frac{1}{2}\beta \right) ^2 + \left(\gamma - \frac{1}{4}\beta^2\right)\text{.}$$
Now, we define $p(\chi)$ so that
$$p(\chi) = \chi^2 + \gamma - \frac{1}{4}\beta^2\text{.}$$
Then, it is easy to see that
$$p(x + \frac{1}{2}\beta) = g(x)\text{.}$$.
Hence, $\alpha$ is a root of $p(x + c)$ for some $c \in F$.
Well, we know that 

    \begin{align*}
        F(\alpha) \cong F[x]\mathbin{/}\left<g(x)\right> \\ &\Leftrightarrow
        F(\alpha) \cong F[x]\mathbin{/}\left<p(x+c)\right> \\ &\Leftrightarrow
        F(\alpha) \cong F[x]\mathbin{/}\left<p(x)\right>
    \end{align*}

So $\alpha$ also is in fact a root of $p(x)$. That is,
$$\alpha^2 + \gamma - \frac{1}{4}\beta^2 = 0$$
$$\alpha^2 = \frac{1}{4}\beta^2 - \gamma$$
$$\alpha = \sqrt{\frac{1}{4}\beta^2 - \gamma}$$
$$\alpha = \sqrt{a} \text{ where } a \in F$$

\end{proof}


%\subsection{Theorem 14.1(ii)}
%Let $f:G \rightarrow H$ be Group homomorphism with $x \in G$.
%Then, $(f(x))^{-1} = f(x^{-1}).$
%
%\begin{proof}
%Observe that it is sufficient to show
%$$f(x^{-1}) * f(x) = e_H$$
%since then,
%    \begin{align*}
%        f(x^{-1}) * f(x) * (f(x))^{-1} &= e_H * (f(x))^{-1} &&\text{(operate each side by $(f(x))^{-1}$)} \\
%        f(x^{-1}) &= (f(x))^{-1}
%    \end{align*}
%
%Well,
%    \begin{align*}
%    f(x^{-1}) * f(x) &= f(x^{-1} * x) && \text{(by homomorphism)} \\
%    &= f(e_G)                                                     \\
%    &= e_H                            && \text{(Theorem 14.1(i)}
%    \end{align*}
%\end{proof}
%
%\subsection{Theorem 14.C.1}
%For group \(G\) and subgroup \(H\), Let \(f:G \to H\) be a homomorphism with kernel \(K\). If \(f\) is injective, then \(K=\{e_G\}\).
%\begin{proof}
%We first show that ($f$ is injective) $\implies$ $K=\{e_G\}$.
%
%Suppose \(x\in K\). Then,
%    \begin{align*}
%        f(x) &= e_H && \text{(by definition of the Kernel)} \\
%        \text{and } f(e_G) &= e_H && \text{(by Thm. 14.1(i)).} \\
%        \text{Hence, } f(x) &= f(e_G) \text{.} \\
%        \therefore x &= e_G && \text{(by definition of Injective)}
%    \end{align*}
%\qedtri
%
%We next show that $K=\{e\} \implies f$ is injective.
%
%Suppose for any $a, b \in G, f(a) = f(b)$.
%Then,
%    \begin{align*}
%        f(a) \cdot (f(b))^{-1} &= e_H && \text{(operate each side by $(f(b))^{-1}$)}  \\
%        f(a) \cdot f(b^{-1}) &= e_H   && \text{(by Thm. 14.1(ii))}    \\
%        f(ab^{-1}) &= e_H             && \text{(by Homomorphism)}     \\
%        ab^{-1} &= e_G                && \text{(since $K = \{e_G\}$)} \\
%        a(b^{-1}b) &= e_G b                                         \\
%        a &= b
%    \end{align*}
%\end{proof}
%
%\subsection{Theorem 22.1}
%Every ideal of $\mathbb{Z}$ is principal.
%
%\begin{proof}
%We will prove the assertion by showing that:
%$$\text{For any ideal } J \subseteq \mathbb{Z} \text{, } \exists n \text{ such that } J = \langle n\rangle \text{.}$$
%Case 1: If $J = \{0\}$, then $J = \langle n\rangle$ where $n = 0$ and we are done.
%
%Case 2: Otherwise, there must exist positive integers in $J$. By the well-ordering property,
%let $n$ be the least positive integer in $J$. We will now show that $J = \langle n\rangle$ which
%will complete the proof:
%\newline
%
%Let $m \in J$. By the division algorithm, we can say that $m = nq + r$ where $0 \le r < n$.
%Since $m, n \in J$, we can say that $m - nq = r$, and $r$ is also an element of $J$.
%Well, because $0 \le r < n$, ($r = 0$) or ($r > 0 \wedge r < n$). The second case is impossible
%because $n$ is the least element in $J$. Therefore, $r = 0$ and $m = nq + 0$.
%\newline
%
%That is, every element of $J$ is a multiple of $n$. In other words, $J = \langle n\rangle$.
%\end{proof}
%
%
%\subsection{Theorem 22.2}
%The only invertible elements of $\mathbb{Z}$ are $1$ and $-1$.
%\begin{proof}
%\end{proof}
%
%\subsection{Theorem 22.3}
%Any two nonzero integers $r$ and $s$ have a greatest common divisor $t$. Moreover,
%$t$ is equal to a linear combination of $r$ and $s$. That is, $$t = kr + ls \text{.}$$
%\begin{proof}
%\end{proof}

\end{document}

